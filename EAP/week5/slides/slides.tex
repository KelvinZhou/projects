\documentclass[10pt,t]{beamer}

% fonts
\usefonttheme{professionalfonts}
\usepackage[osf, sc]{mathpazo} % for rm
\usepackage{courier} % for mono
% \usepackage[euler-digits]{eulervm}
\usefonttheme{serif}
% more about eulervm:
% \mathrm -> rmdefault
% \mathbf -> bfdefault
% \mathsf -> sfdefault
% \mathtt -> ttdefault
% Thus, you should redefine the default text fonts before loading the eulervm package!
% redefine: \mathit, \mathbold, \mathcal

%math
\usepackage{amsmath}
\DeclareMathOperator*{\argmin}{arg\,min}
\DeclareMathOperator*{\argmax}{arg\,max}
\newcommand*\diff{\mathop{}\!\mathrm{d}}
% color
\definecolor{brown}{rgb}{0.4, 0.208, 0.192}
\definecolor{gray1}{rgb}{0.125, 0.125, 0.125}
\definecolor{gray2}{rgb}{0.157, 0.157, 0.157}
\definecolor{green2}{rgb}{0.16862745098039217,0.4,0.19607843137254902}
\definecolor{blue2}{rgb}{0,0.40784313725490196,0.6352941176470588}
\definecolor{brown2}{rgb}{0.6235294117647059,0.2901960784313726,0}

\definecolor{blue}{rgb}{0.18, 0.2, 0.529}
\definecolor{orange}{RGB}{206,117,59} % used for strong
\definecolor{green}{RGB}{49,104,108} % used for everywhere, link,...
\definecolor{red}{RGB}{133,60,66}
\definecolor{pink}{RGB}{184,86,150}



\setbeamercolor{caption name}{fg=green}

% settings for itemize and enumerate
\setbeamertemplate{itemize item}{\textbullet}
\setbeamertemplate{itemize subitem}{\bfseries --}
\setbeamertemplate{itemize subsubitem}{$\circ$}
\setbeamercolor{itemize item}{fg=green}
\setbeamercolor{itemize subitem}{fg=brown}
\setbeamercolor{itemize subsubitem}{fg=gray2}

\setbeamertemplate{enumerate item}{\arabic{enumi}.}
\setbeamertemplate{enumerate subitem}{(\roman{enumii})}
\setbeamertemplate{enumerate subsubitem}{\alph{enumiii}.}
\setbeamercolor{enumerate item}{fg=green}
\setbeamercolor{enumerate subitem}{fg=brown}
\setbeamercolor{enumerate subsubitem}{fg=gray2}
% main page style
\setbeamercolor{background canvas}{bg=white}
\setbeamercolor{normal text}{fg=black}
\setbeamertemplate{footline}[frame number]
\setbeamertemplate{navigation symbols}{}
\setbeamercolor{block title}{bg=green!90!black,fg=white}
\setbeamercolor{block body}{bg=green!10}
% frametitle
\setbeamertemplate{frametitle}
{
    \vskip 5pt
    \strut\Large\textcolor{green}{\insertframetitle}\strut
    \ifx\insertframesubtitle\empty
    \vskip-1.7ex
    \textcolor{green}{\hrule height0pt depth0.5pt \relax}
    \else
    \vskip-1ex
    \strut\small\textcolor{green}{\insertframesubtitle}\strut
    \textcolor{green}{\hrule height0pt depth0.5pt \relax}
    \fi
}
% titlepage
\setbeamertemplate{title page}
{
  \makebox[\textwidth][c]{
    \begin{minipage}[b][0.75\paperheight][c]{0.9\textwidth}
      \centering
      \huge\rm\inserttitle\\
      \Large\rm\insertsubtitle
      \vskip-1.5ex
      \textcolor{green}{\hrule height0pt depth0.8pt \relax}
      \vskip5pt
      \rm\normalsize\insertauthor
    \end{minipage}
  }

  \makebox[\textwidth][c]{
    \begin{minipage}[b][0.25\paperheight][c]{0.4\textwidth}
      \centering
      \rm\insertinstitute
      \par
      \rm\insertdate
    \end{minipage}
  }
}

\usepackage{hyperref}

% dashed underline
\usepackage{ulem}

\usepackage{booktabs}
% ********************************************************************
% tikz
% ********************************************************************
\usepackage{tikz}
\usetikzlibrary{positioning}
\usetikzlibrary{decorations.markings}
\usepackage{pgfplots}






\title{Empirical Asset Pricing \\ Problem Set $5$}
\author{Yu Zhou, HKUST}
\date{\today}


\begin{document}

\maketitle

\begin{frame}{Apply GMM to the CX of stock returns}
\begin{itemize}
	\item In this problem set, we are going to learn how to apply GMM to the cross-section of stock returns (in particular, 25 portfolios).
  \item Data from Kenneth French's website:
  \begin{itemize}
    \item value-weighted monthly returns on the 25 portfolios sorted based on size and the book-to-market ratio;
    \item factor returns.
  \end{itemize}
\end{itemize}
\end{frame}


\begin{frame}{Q1: Size effect and value effect}
\begin{table}
\begin{tabular}{lccccc}
\toprule
 & Small & ME2 & ME3 & ME4 & Large \\
\cmidrule{1-6}
Growth & $0.59$ & $0.65$ & $0.73$ & $0.75$ & $0.67$\\
& $1.64$ & $2.74$ & $3.34$ & $4.04$ & $4.23$ \\
BM2 & $0.7$ & $0.93$ & $0.91$ & $0.77$ & $0.63$\\
& $2.41$ & $4.18$ & $4.72$ & $4.25$ & $4.03$ \\
BM3 & $0.98$ & $0.97$ & $0.9$ & $0.85$ & $0.7$\\
& $3.67$ & $4.49$ & $4.68$ & $4.46$ & $4.21$ \\
BM4 & $1.13$ & $1.02$ & $1.10$ & $0.93$ & $0.64$\\
& $4.59$ & $4.63$ & $4.91$ & $4.58$ & $3.26$ \\
Value & $1.31$ & $1.21$ & $1.07$ & $1.01$ & $0.91$\\
& $4.77$ & $4.67$ & $4.24$ & $3.91$ & $3.6$ \\
\bottomrule
\end{tabular}
\caption{The value effect exist. However, for the low book-to-market portfolios (i.e., Growth and BM2), the size effect does not exist.}
\end{table}
\end{frame}


\begin{frame}{Q2: Time-series regression}
\begin{itemize}
  \item For each portfolio $i$, run OLS regressions
  $$
  R_{it}^e = \alpha_i + \beta_i RMRF_t + \varepsilon_{it}
  $$
\end{itemize}
\begin{table}
\begin{tabular}{lccccc}
\toprule
 & Small & ME2 & ME3 & ME4 & Large \\
\cmidrule{1-6}
Growth & $-0.5$ & $-0.21$ & $-0.11$ & $0.01$ & $0.03$\\
& $-2.32$ & $-1.76$ & $-1.19$ & $0.22$ & $0.59$ \\
BM2 & $-0.24$ & $0.11$ & $0.15$ & $0.04$ & $-0.01$\\
& $-1.5$ & $1.08$ & $2.19$ & $0.71$ & $-0.14$ \\
BM3 & $0.06$ & $0.16$ & $0.15$ & $0.1$ & $0.05$\\
& $0.4$ & $1.79$ & $2.09$ & $1.48$ & $0.78$ \\
BM4 & $0.28$ & $0.21$ & $0.22$ & $0.15$ & $-0.1$\\
& $2.25$ & $2.13$ & $2.74$ & $1.84$ & $-1.18$ \\
Value & $0.39$ & $0.28$ & $0.14$ & $0.06$ & $0.03$\\
& $2.69$ & $2.19$ & $1.24$ & $0.47$ & $0.24$ \\
\bottomrule
\end{tabular}
\caption{The estimated $\alpha_i$ and associated OLS t-statistics. The CAPM does not explain the value effect. $\alpha$ is larger for higher book-to-market stocks.}
\end{table}
\end{frame}

\begin{frame}{Q2: Time-series regression (cont'd)}
\begin{table}
\begin{tabular}{lccccc}
\toprule
 & Small & ME2 & ME3 & ME4 & Large \\
\cmidrule{1-6}
Growth & $1.62$ & $1.27$ & $1.25$ & $1.09$ & $0.96$\\
& $11.67$ & $24.69$ & $36.23$ & $49.63$ & $66.62$ \\
BM2 & $1.40$ & $1.23$ & $1.13$ & $1.08$ & $0.95$\\
& $14.9$ & $24.49$ & $55.23$ & $51.2$ & $74.87$ \\
BM3 & $1.37$ & $1.2$ & $1.12$ & $1.11$ & $0.97$\\
& $19.77$ & $26.09$ & $35.96$ & $31.21$ & $32.87$ \\
BM4 & $1.27$ & $1.21$ & $1.17$ & $1.16$ & $1.10$\\
& $17.77$ & $23.81$ & $25.75$ & $25.74$ & $26.97$ \\
Value & $1.37$ & $1.38$ & $1.38$ & $1.41$ & $1.31$\\
& $16.1$ & $22.01$ & $25.27$ & $21.27$ & $18.13$ \\
\bottomrule
\end{tabular}
\caption{The estimated $\beta_i$ and associated OLS t-statistics. The CAPM explain the size effect within BM3, BM4, and Value portfolio.}
\end{table}
\end{frame}

\begin{frame}{Q2: Time-series regression (cont'd)}
\begin{itemize}
  \item GRS test statistic: $83.49$ with $p$-value $0$.
  \item Use GMM to conduct tests:
  \begin{itemize}
    \item moment conditions:
    $$
    E[(1 \ RMRF)' \otimes \varepsilon] = 0
    $$
    \item \# moments = \# params = 50 (exactly identified) $\Rightarrow$ $A = I_{50}$.
    \item $\partial \text{moment conditions} / \partial b'$:
    $$
    \underbrace{D}_{50\times50} = - \left[\begin{array}{cc} 1 & E(RMRF) \\ E(RMRF) & E(RMRF^2) \end{array}\right] \otimes I_{25}
    $$
    \item F test statistic: $\hat{\alpha}' cov(\hat{\alpha})^{-1} \hat{\alpha} \sim \chi^2_{25}$
    \item F test statistic: $83.49$ with $p$-value $0$. It is extremely similar to the GRS test statistic.
  \end{itemize}
  \item These tests reject the CAPM.
\end{itemize}
\end{frame}

\begin{frame}{Q2: Time-series regression (cont'd)}
\begin{itemize}
  \item For each book-to-market quintile, pick the portfolio with smallest stocks and largest stocks, to investigate whether the difference in betas are statistically significant or not.
\begin{table}
\begin{tabular}{lccccc}
\toprule
 & Growth & BM2 & BM3 & BM4 & Value \\
\cmidrule{1-6}
$\gamma'\beta$ & $-0.66$ & $-0.46$ & $-0.4$ & $-0.16$ & $-0.07$ \\
$t$-stat & $-13.35$ & $-11.88$ & $-12.56$ & $-5.73$ & $-1.94$\\
\bottomrule
\end{tabular}
\caption{For every B/M quintile, small stocks have larger market betas than large stocks. These difference is significant except for the value stocks.}
\end{table}
\end{itemize}
\end{frame}


\begin{frame}{Q3: Cross-sectional regression}
\begin{itemize}
  \item Run a cross-sectional regression of average excess returns on the CAPM betas which have been estimated in time-series regressions.
  $$
  E_T[R^e_{it}] = \lambda \beta_i + \alpha_i
  $$
  \item The OLS estimates for $\lambda$ is $\hat{\lambda} = 0.72$, and the associated $t$-statistic is $20.56$. This $t$-statistic is high. One reason is that the estimation errors on $\beta_i$ have been ignored.
\end{itemize}
\end{frame}

\begin{frame}{Q3: Cross-sectional regression (cont'd)}
\begin{itemize}
  \item Run the Fama-MacBeth regressions for every month to obtain the monthly $\lambda_t$.
  $$
  R_{it}^e = \lambda_t \beta_i + \alpha_{it}
  $$
  \item the average factor risk premium is $\hat{\lambda}_{FMB} = \frac{1}{T}\sum \hat{\lambda}_t = 0.72$, and the associated $t$-statistic is $4.21$ which is much lower than the OLS $t$-statistic. This may because FMB take considerations of the correlation among the error terms.
  \item To see if $\alpha$ is too large to not, we use $\hat{\alpha}'cov(\hat{\alpha})^{-1}\hat{\alpha} \sim \chi^2_{N-1}$.
  \begin{itemize}
    \item the F statistic is $83.71$ with $p$ value of $0$.
    \item CAPM is rejected.
  \end{itemize}
\end{itemize}
\end{frame}

\begin{frame}{Q3: Cross-sectional regression (cont'd)}
\begin{itemize}
  \item Run the Fama-MacBeth regressions with intercepts.
  \begin{itemize}
    \item $\hat{\lambda}_0 = 0.39$, $t(\hat{\lambda}_0) = 1.22$;
    \item $\hat{\lambda}_1 = 0.40$, $t(\hat{\lambda}_1) = 1.14$.
  \end{itemize}
  \item $\lambda_0$ is the excess return of zero-beta asset and $\lambda_1$ is the premium of market risk factor. If CAPM is the true asset pricing model, the $\lambda_0$ should be $0$.
\end{itemize}
\end{frame}


\begin{frame}{Q3: Cross-sectional regression (cont'd)}
GMM
\begin{itemize}
  \item We have 75 moment conditions and 51 parameters to be estimated. The GMM is overidentified.
  \item $\hat{\lambda} = 0.72$.
  \item D matrix:
  $$
  D = - \left[\begin{array}{ccc} I_N & E(RMRF)I_N & 0 \\
  E(RMRF)I_N & E(RMRF^2)I_N & 0 \\
  0 & \lambda I_N & \beta\end{array}\right]
  $$
  \item A matrix:
  $$
  A = \left[\begin{array}{cc} I_{2N} & 0 \\
  0 & \beta' \end{array}\right]
  $$
  \item The $t$-statistic is $4.43$ which is not equal but quite similar to the FMB $t$-statistic.
\end{itemize}
\end{frame}

\begin{frame}{Q3: Cross-sectional regression (cont'd)}
\begin{itemize}
  \item Pricing errors under the cross-sectional regressions:
\begin{table}
\begin{tabular}{lccccc}
\toprule
 & Small & ME2 & ME3 & ME4 & Large \\
\cmidrule{1-6}
Growth & $-0.57$ & $-0.26$ & $-0.16$ & $-0.03$ & $-0.01$\\
& $-1.57$ & $-1.09$ & $-0.72$ & $-0.17$ & $-0.08$ \\
BM2 & $-0.3$ & $0.05$ & $0.1$ & $-0.01$ & $-0.05$\\
& $-1.04$ & $0.24$ & $0.53$ & $-0.03$ & $-0.29$ \\
BM3 & $0$ & $0.11$ & $0.1$ & $0.05$ & $0.01$\\
& $-0.01$ & $0.5$ & $0.52$ & $0.27$ & $0.05$ \\
BM4 & $0.23$ & $0.16$ & $0.18$ & $0.1$ & $-0.15$\\
& $0.92$ & $0.7$ & $0.85$ & $0.5$ & $-0.77$ \\
Value & $0.33$ & $0.22$ & $0.09$ & $0$ & $-0.02$\\
& $1.2$ & $0.85$ & $0.34$ & $-0.01$ & $-0.08$ \\
\bottomrule
\end{tabular}
\caption{Like the time-series regressions, the CAPM still does not fully explain the value effect under the cross-sectional regressions. $\alpha$ is larger for higher book-to-market stocks.}
\end{table}
\item Overidentification test statistic $78.96$ with $p$-value of $0$. So CAMP is rejected.
\end{itemize}
\end{frame}

\begin{frame}{Q3: Cross-sectional regression (cont'd)}
\begin{itemize}
  \item As we have seen, the differences of $t$-statistic and $\chi^2$-statistic between GMM and FMB are small.
  \item These differences reflect the fact that GMM has taken the estimation errors of $\hat{\beta}$ into consideration; while FMB treats the $\hat{\beta}$ estimated in the first-stage TS regressions as the true beta.
  \item In this example, the differences are small because portfolios' betas are estimated with less errors.
\end{itemize}
\end{frame}

\begin{frame}{Q3: Cross-sectional regression (cont'd)}
\begin{table}
\begin{tabular}{lcccc}
\toprule
 & CX & FMB & GMM & TS \\
\cmidrule{1-5}
$\hat{\lambda}$ & $0.72$ & $0.72$ & $0.72$ & $0.67$\\
$t(\hat{\lambda})$ & $20.56$ & $4.21$ & $4.43$ & $4.23$\\
\bottomrule
\end{tabular}
\caption{The point estimate of the market price of risk is smaller for TS regressions which is the average excess return of the market portfolio. The associated $t$-statistics are similar for FMB, GMM and TS but not for CX which implies that the cross-sectional correlation in monthly returns across assets matters a lot.}
\end{table}
\end{frame}


\begin{frame}{Q4: Allow for serial correlation in returns}
\begin{itemize}
  \item Serial correlation in returns is reasonable when the assets have some ``stale'' pricing.
  \item Allowing for serial correlation in monthly returns in the form of 12-month lags with Newey-West weights.
  \item $\hat{\lambda} = 0.72$, $t_{NW}(\hat{\lambda}) = 4.18$ which is slightly lower than t-statistic in no serial correlation case.
\end{itemize}
\end{frame}











\end{document}

